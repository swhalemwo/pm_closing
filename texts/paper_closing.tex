% Created 2024-04-07 zo 14:03
% Intended LaTeX compiler: pdflatex
\documentclass[12pt]{article}

\usepackage[hyphens]{url}                
\usepackage{hyperref}
\usepackage[hyphenbreaks]{breakurl}
\usepackage{rotating}
\usepackage{wrapfig}
\usepackage{pdflscape}
\usepackage{fixltx2e}
\usepackage{graphicx}
\usepackage{amsmath}
\usepackage{amsfonts}
\usepackage[section]{placeins}
\usepackage{dirtree}
\usepackage{siunitx}
\usepackage{afterpage}
\usepackage{pdflscape}
\usepackage{svg}


\usepackage{booktabs}
\usepackage{dcolumn}

\usepackage{bibentry}

\sisetup{detect-all}

\sloppy
\usepackage{scalerel,stackengine}

\stackMath
\newcommand\reallywidehat[1]{%
\savestack{\tmpbox}{\stretchto{%
  \scaleto{%
    \scalerel*[\widthof{\ensuremath{#1}}]{\kern-.6pt\bigwedge\kern-.6pt}%
    {\rule[-\textheight/2]{1ex}{\textheight}}%WIDTH-LIMITED BIG WEDGE
  }{\textheight}% 
}{0.5ex}}%
\stackon[1pt]{#1}{\tmpbox}%
}

\usepackage{caption}
\usepackage[draft]{todonotes}

\captionsetup{skip=0pt}
\usepackage[utf8]{inputenc}
\usepackage[style=apa, backend=biber]{biblatex} 
\usepackage[english, american]{babel}
\DeclareLanguageMapping{american}{american-apa}
\DeclareFieldFormat{apacase}{#1}

\usepackage[T1]{fontenc}
\usepackage{csquotes}

\addbibresource{/home/johannes/Dropbox/references.bib}
\addbibresource{/home/johannes/Dropbox/references2.bib}

\usepackage{floatrow}

\usepackage{listings}
\usepackage{xcolor}
\usepackage{colortbl}

\lstset{
  language=R,                    
  basicstyle=\footnotesize,      
  numbers=left,                  
  numberstyle=\tiny\color{gray}, 
  stepnumber=1,                  
  numbersep=5pt,                 
  backgroundcolor=\color{white}, 
  showspaces=false,              
  showstringspaces=false,        
  showtabs=false,                
  frame=single,                  
  rulecolor=\color{black},       
  tabsize=2,                     
  captionpos=b,                  
  breaklines=true,               
  breakatwhitespace=false,       
  title=\lstname,                
  keywordstyle=\color{red},     
  commentstyle=\color{blue},  
  stringstyle=\color{violet},     
  escapeinside={\%*}{*)},        
  morekeywords={*,...}           
} 




% \usepackage{crimson}
% \usepackage{microtype}

% \usepackage{helvet}
% \renewcommand{\familydefault}{\sfdefault}

\usepackage{tgtermes} % times font


\usepackage{fancyhdr}
\usepackage{setspace}
\onehalfspacing
\usepackage{longtable}
\usepackage{subfig}
% \usepackage[a4paper, total={18cm, 24cm}]{geometry}
\usepackage[a4paper, margin=2.5cm]{geometry}

\pagestyle{fancy}
\fancyhf{}
\renewcommand{\headrulewidth}{0pt}
\renewcommand{\maketitle}{}

\usepackage{enumitem}
\setlist[itemize]{topsep=0pt,itemsep=0pt,parsep=0pt,partopsep=0pt}

\usepackage{multicol}
\setlength\multicolsep{0pt}



\newlist{propertyList}{itemize}{1}
\setlist[propertyList]{
  label=\textbullet,
  noitemsep,
  leftmargin=10pt,
  before=\begin{multicols}{3},
  after=\end{multicols}
  }

% \cfoot {Johannes Aengenheyster}
\rfoot {\thepage}

\listfiles

\setlength{\parindent}{1.2cm}
\author{Johannes }
\date{\today}
\title{}
\hypersetup{
 pdfauthor={Johannes },
 pdftitle={},
 pdfkeywords={},
 pdfsubject={},
 pdfcreator={Emacs 29.2 (Org mode 9.6.21)}, 
 pdflang={English}}
\begin{document}




\section*{Identity and the closure of private art museums}



\subsection*{Introduction}


In recent years, private collector-run museums have seen an unprecedented proliferation, with hundreds of organizations founded since the beginning of the millenium \parencite{Velthuis_etal_2023_boom,LarrysList_2015_report}.
These museums, run by individual collectors, constitute an addition to established organizational forms of art provision such as museums operated and financed by the state, foundations or associations.
While the emergence of this organizational form has attracted attention (cf. \cite{Kolbe_etal_2022_privatemuseum} for a literature review), the process of closure has received considerably less attention.
In this paper, I address this gap by investigating the closure event of members of this new organizational form.
This exercise does not only provide insights into private museums in particular, but also more generally into processes of organizational identity, especially regarding legitimation and reputation. 


A number of findings highlight the role of identity in the process of private museum closure. 
First, I find that museums which are recognized by an international museum database are less likely to close than those which are not.
Secondly, the results furthermore indicate that museums with more ambiguous names are more likely to close.
Both these findings possibly indicate that adherence to museum standards enhances legitimacy, which in turn improves survival prospects.
Finally, founder reputation, in particular its loss, is found to be substantial as museums whose founders are dropped from a collector ranking are more likely to close than museums whose founders both are and are not included.

\bigbreak
\noindent
\textbf{keywords}: museum, closure, identity

\subsection*{Theoretical framework}



\subsubsection*{Identity and the survival prospects of non-profit organizations}

An identity that is legitimate (understandable) and favorable (positive) has been shown to have a vast number of positive outcomes for organizations \parencite{Lange_Lee_Dai_2010_reputation}.
As gaining the approval of different stakeholders is necessary for being understood and/or supported by other organizations in the environment, having a clearly defined and favorable identity influences resource acquisition and thereby survival prospects \parencite{Rao_1994_reputation}.
Even for non-profit organizations such as museums, which do not produce tangible goods (a domain where identity-related effects are present, e.g. \cite{Hsu_2015_granted,Bogaert_etal_2014_ecological}), a clear and positive identity might be helpful in a number of ways:
In particular in the context of private museums, decline of legitimacy or reputation could lead to lower visitor numbers, less discounts from art dealers, less government grants or lower chances on collaborations with other museums or corporations, resulting in lower revenues and higher acquisition costs.


The empirical record of the link between identity and survival for non-profit organizations is mixed: 
For example \textcite{Bielefeld_1994_survival} finds that NPOs who pursue less legitimation strategies (obtaining endorsements, lobbying or contributing to local causes) are more likely to close.
However, other studies find limited impact of legitimacy on survival, both when measured via density \parencite{Bogaert_etal_2014_ecological} or from archival sources and interviews \parencite{Fernandez_2007_dissolution}.
Thus, even while this study cannot conclusively answer the overall link between idenity and survival, it nevertheless contributes insights into which dimensions of organizational identity do, and which do not contribute to survival in a particular organizational population; namely that of private art museums.


Legitimacy might be obtained by isomorphism, i.e. adopting features associated with blueprints \parencite{diMaggio_1983_iron}.
Understandability \parencite{Glynn_Abzug_2002_names} might thus be limited if an organization is atypical \parencite{Rosch_1975_family}, i.e. displays an unusual combination of features.
In the case of private museums, such cases might be combinations of features from house museums (e.g. accessibility only by appointment) with full-service museums (e.g. extensive activities). 
Next to "hard" organizational features, adherence to naming conventions has been found to enhance legitimacy \parencite{Glynn_Abzug_2002_names}; organizations that adhere to field norms about name length, name ambiguity (usage of artificial names) and name domain specificity (mentioning industry) were judged more legitimate. 
However, the category of museums could be relatively flexible (for example, it is generally not subject to state regulation), which might result in a relatively high tolerance of diversity as "anything goes" and hence limited "devaluation" of non-conforming, atypical members \parencite{Bogaert_etal_2014_ecological}.


Furthermore, identity of a private museum may be influenced by the identity of persons most closely affiliated with it, in particular the identity of its founders.
Given that founders are the very reason behind the existence of these organizations, and sometimes plan these as their personal legacies \parencite{Walker_2019_collector}, founder status gains or losses may very well reflect on the wider organization and affect its survival propspects.


\bigbreak
\noindent
\textbf{Hypothesis 1}: Private museums with identities more understandable, legitimate and favorable are less likely to close. 


\subsubsection*{Other predictors of organizational survival}


\paragraph*{Age}



The age of an organization is considered in different ways to be related to closing.
What started with liability of newness \parencite{Stinchcombe_1965_structure}, which posited that organizational mortality decreases over time due to acquisition of capabilities and connections, has lead to a number of extensions \parencite{Carroll_Khessina_2019_demography,Hannan_1998_mortality}:
Liability of adolescence argues that shortly after foundings organizations can still draw on endowments, the expiration of which leads to a peak of mortality some time after opening.
While according to this view mortality still decreases after the peak as the organization becomes more established, other perspectives argue also for a positive relationship between age and mortality:
According to the liability of obsolescence paradigm the environment changes faster than organizations which are seen as relatively constrained by inertia, which therefore leads to declining fitness and survival chances.
Similar, according to "liability of senescence" capabilities/routines/coalitions can constrain the actions of organizations.
Given such conflicting predictions, \textcite{Carroll_Khessina_2019_demography} propose to model age dependence with piece-wise constant hazard models (rather than parametric models) to estimate age-specific effects which can, but do not have to, indicate a relationship between age and mortality.



\paragraph*{Organizational transformation}

\textcite{Carroll_Khessina_2019_demography} argue that transformation of core features of organizations such as changes in technology or authority can have divergent consequences. 
On the one hand, it can disrupt internal routines as well as external customer relations by decreasing (perceived) reliability and accountability as the change upsets established perceptions, which can lead to higher mortality. 
On the other hand, organizational transformation can be necessary to adapt to a changing environment, and hence be beneficial for survival.



In the case of private museums, the death of the founder might constitute a substantial transformation as authority has to be reconfigured.
Abandonment by insiders (such as divorce of directors) has been argued to contribute to NPO closure \parencite{Duckles_Hager_Galaskiewicz_2005_close}.\footnote{Conflict is similarly argued to contribute to closure but is not testable with the current data.}
Even if plans have been made for a handover, the new leaders might not share the same commitment to art as the original founder, potentially decreasing museum sustainability. 
In the case of private museums, founder death has been speculated to pose a challenge to their sustainability as "seldom do heirs share a similar passion or wish to take on the financial burden of maintaining private museum indefinitely" \parencite[p.234]{Walker_2019_collector}.
However, existing research has not found a straightforward effect of founder death on museum closure \parencite{Velthuis_Gera_forthcoming_fragility,Velthuis_etal_2023_boom}.
Nevertheless as previous research has relied primarily on descriptive statistics, investigating founder death in a multivariate survival model allows to investigate the effect of founder death with higher precision.
\bigbreak
\noindent





\subsection*{Data and Methods}


\subsubsection*{Dependent Variable}

A newly constructed database, assembled from different sources,  is used to measure the survival of private museums (cf. \textcite{Velthuis_etal_2023_boom} for details).
First, existing databases of private art museums are harmonized.
Secondly, the online presences of 14 English art publications were exhaustively searched for any mention of private museums.
After identifying unique institutions, data on these organizations was collected such as opening year, closing year (of those which closed) and founder information.
Currently, the database includes 547 museums (of which 446 are currently open, the remainder having closed or transformed into other organizational forms).
Using this database allows to reconstruct the lifecourse of the each institution with the the unit of analysis as museum-year:
The main dependenent variable is thus an indicator of museum closure, which takes the value of 1 for the museums which close in the year of their closure, and 0 otherwise.
Typically for survival data, museums that do not close during the observation period are right-censored, i.e. for such museums the dependent variable is zero for all years.


\subsubsection*{Main Predictors: Identity Measures}


The organizational identity of a museum is measured primarily in terms of its legitimacy, i.e. its adherence to museum standards, and reputation of its founders.

\paragraph*{Legitimacy}


Two measures are constructed that aim to capture an effect of legitimacy on closing.
First, self-identification measures whether a the name of a museum includes the term "Museum", "Foundation", "Collection", or other (such as "Villa", "Institute", "Center" or "Gallery", which are grouped together as they are less frequent).
Since the name is one of the first aspects that stakeholders perceive when engaging with an organization, it might very well shape expectations and influence forms of engagement.
In particular, it might be that museums that also refer to themselves as museums are seen as more favorable and worthy of visits, collaborations and support than other museums.


Secondly, a binary predictor measures whether the museum is included in the "Museums of the World" (MOW) database \parencite{deGruyter_2021_MOW}, a database containing over 55.000 museums.
While the exact criteria of the research process leading to the MOW database are not known, it is plausible that museums who correspond to an ideal-typical museum are more likely to be included.
Assuming that museums are seen similarly by the creators of the MOW database and other audiences, inclusion in the MOW database constitutes a proxy for adherence to museum standards. 


\paragraph*{Reputation}


Reputation is measured via inclusion of the founder in the Artnews top 200 collector ranking \parencite{Artnews_ranking}.
This ranking, established in 1990, lists each year the 200 collectors the editors of the magazine consider most relevant.
For each year, a museums can have one of three values: "Not included"; in which case the founder is not included in the year in question and has not been so in the past, "Included"; when the founder is included, or "Dropped", for cases where the founder was included at some point in the past but is no longer included in the current year.
Cases where multiple individuals are involved in the founding of a private museum are resolved by aggregating the Artnews ranking inclusion history to the founder founder level (i.e., if one of two founders is dropped from the Artnews ranking but the other remains included, the founder couple as a whole is considered included).





\subsubsection*{Control variables}


\bigbreak
\noindent
A number of control variables are used: 
On the level of the founder, gender constitutes a time-invariant variable; with possible values being male, female and couple.
Organizational transformation, such as change of authority has been to increase chances of closure both generally \parencite{Carroll_Khessina_2019_demography} and in the case of private museums specifically (\cite[p.234]{Walker_2019_collector}: "seldom do heirs share a similar passion or wish to take on the financial burden of maintaining private museum indefinitely").
Therefore, a measure of founder death is constructed as a time-varying covariate, and is set to 1 for museum-years after the death of the founder. 
Centrality of the founder's vision is measured via the proxy of whether the name of the founder is included in the name of the museum.


Furthermore, organizational theory \parencite{hannan89_organ} predicts a U-shaped relationship between density and mortality:
While at low levels, increases in the number of organizations are argued to primarily increase legitimacy and lower exit rates, increases at higher numbers increase the competition over resources and thereby raise mortality rates.
On the level of the environment, therefore the density (number) of private museum per country is included both in linear and squared form.








\subsubsection*{Analytical framework}



First, we investigate the relationship between age and closing descriptively using hazard rates and the Kaplan-Meier survival function.
This non-parametric method estimates the hazard rate as the conditional probability of exiting at time t\textsubscript{i} given being alive as \(h(t_i) = \frac{d_i}{n_i}\), (with \(n_i\) as the number of units at risk at time \(t_i\) and \(d_i\) as the number of exits at time \(t_i\)), and the survival function as the product of these conditional probabilities as \(S(t) = \prod_{t_i \geq t} \left(1-h(t_i) \right)\).



Secondly, we use Cox proportional hazard models which estimate the relationship between covariates and the hazard rate.
The Cox-proportional hazard model has the form \(h(t,\mathbf{x}) = h_0(t) \psi\), where \(\psi = \exp(\sum_{j} \mathbf{x}_j \beta_j)\); with \(h(t)\) as the hazard rate, \(h_0(t)\) as the baseline hazard rate, \(\mathbf{x}\) as the set of variables indexed by \(j\), and \(\beta\) as the coefficient vector.
Partial maximum likelihood estimation allows calculating \(\beta\) without defining the baseline hazard rate, which therefore is not constrained to a parametric form. 
Coefficients are interpeted as log hazard ratios, i.e. as multipliers of the baseline hazard rate. 



\subsection*{Results}


\subsubsection*{Summary statistics}


% latex table generated in R 4.3.3 by xtable 1.8-4 package
% Sun Apr  7 12:05:30 2024
\begin{table}[ht]
\centering
\begin{tabular}{llrrrrrr}
  \hline
 & & \multicolumn{2}{c}{Museum} & \multicolumn{4}{c}{Museum-year} \\ 
\cmidrule(r){3-4}\cmidrule(r){5-8} \multicolumn{1}{l}{} & \multicolumn{1}{l}{Variable} & \multicolumn{1}{l}{Count} & \multicolumn{1}{l}{Mean} & \multicolumn{1}{l}{Mean} & \multicolumn{1}{l}{SD} & \multicolumn{1}{l}{Min.} & \multicolumn{1}{l}{Max.}\\ 
 \hline
  \multicolumn{8}{l}{\textbf{Founder}} \\ 
 & Gender - Male & 275 & 0.574 &   0.593 &   0.49 & 0 & 1 \\ 
   & Gender - Female & 76 & 0.159 &   0.169 &   0.37 & 0 & 1 \\ 
   & Gender - Couple & 128 & 0.267 &   0.238 &   0.43 & 0 & 1 \\ 
   & Founder died & 58 & 0.121 &   0.098 &   0.30 & 0 & 1 \\ 
   \multicolumn{8}{l}{\textbf{Museum}} \\ 
 & Self-Identification - Museum & 178 & 0.372 &   0.410 &   0.49 & 0 & 1 \\ 
   & Self-Identification - Foundation & 101 & 0.211 &   0.189 &   0.39 & 0 & 1 \\ 
   & Self-Identification - Collection & 74 & 0.154 &   0.140 &   0.35 & 0 & 1 \\ 
   & Self-Identification - Other & 126 & 0.263 &   0.261 &   0.44 & 0 & 1 \\ 
   & Founder name in Museum name & 229 & 0.478 &   0.488 &   0.50 & 0 & 1 \\ 
   & MOW inclusion & 91 & 0.190 &   0.304 &   0.46 & 0 & 1 \\ 
   & AN Ranking - Not Included &  &  &   0.797 &   0.40 & 0 & 1 \\ 
   & AN Ranking - Included &  &  &   0.150 &   0.36 & 0 & 1 \\ 
   & AN Ranking - Dropped &  &  &   0.054 &   0.23 & 0 & 1 \\ 
   \multicolumn{8}{l}{\textbf{Environment}} \\ 
 & PM density (country) &  &  &   0.411 &   1.23 & 0.00077 & 27.26 \\ 
   & Pop. (millions) within 10km &  &  &   1.322 &   1.69 & 0.000087 & 10.81 \\ 
   & Pop. (millions) country &  &  & 156.197 & 287.82 & 0.037 & 1412.36 \\ 
   & Nbr PM within 10km &  &  &   1.601 &   2.97 & 0 & 15 \\ 
   & PM density (10km) &  &  &   1.423 &   4.01 & 0 & 41.85 \\ 
   & Europe and North America & 312 & 0.651 &   0.645 &   0.48 & 0 & 1 \\ 
   & Region - Africa & 7 & 0.015 &   0.012 &   0.11 & 0 & 1 \\ 
   & Region - Asia & 128 & 0.267 &   0.282 &   0.45 & 0 & 1 \\ 
   & Region - Europe & 241 & 0.503 &   0.501 &   0.50 & 0 & 1 \\ 
   & Region - Latin America & 22 & 0.046 &   0.042 &   0.20 & 0 & 1 \\ 
   & Region - North America & 71 & 0.148 &   0.143 &   0.35 & 0 & 1 \\ 
   & Region - Oceania & 10 & 0.021 &   0.019 &   0.14 & 0 & 1 \\ 
   \hline
\end{tabular}
\caption{Summary Statistics} 
\label{tbl:t_sumstats}
\end{table}

In total, 479 private museums are included in the database, 53 of them have closed.
6903 museum-years are observed; table \ref{tbl:t_sumstats} shows summary statistics for all variables.

\subsubsection*{Hazard rate and Age Dependence}


\begin{figure}[htbp]
\centering
\includegraphics[width=16cm]{../figures/p_hazard.pdf}
\caption{\label{fig:p_hazard}Private Museum hazard function}
\end{figure}

\begin{figure}[htbp]
\centering
\includegraphics[width=14cm]{../figures/p_surv.pdf}
\caption{\label{fig:p_surv}Private Museum Survival probability}
\end{figure}


Figure \ref{fig:p_hazard} shows the hazard rate over time, figure \ref{fig:p_surv} the corresponding survival probability.
The hazard rate doubles over the first 10 years from approximately 0.45\% to 0.9\%, at which value it approximately stays constant for the next 20 years, after which it appears to decline.
However, the decline after thirty years is presumably less certain as very few private museums have already reached this age.


The increase in the first years is likely due to endowments: As the establishment of a museum requires considerable resources, their founders appear to allocate enough resources to keep the risk of closure small in the first years after opening.
Given that founders are likely aware that their endeavor is unlikely to generate a profit, they are not vulnerable to the liability of newness \parencite{Stinchcombe_1965_structure}.
The relative stability of the hazard rate after ten years has no correspondence in classical theories on age dependence:
While an increase in mortality as observed here is predicted by the idea of "liability of adolescence" \parencite{Carroll_Khessina_2019_demography}, this framework also predicts a decline in mortality after a peak due to becoming established.
Furthermore, while "liability of obsolescence" and "liability of senescence" (ibid.) predict higher mortality for older organizations due to mismatch with the environment and inertia (inflexible internal routines), it seems questionable whether museums would have reached obsolescence or senescence already after ten years.
Furthermore, the applicability of obsolescence and senescence might be limited for non-profit organizations.


Overall the entire life span, the average hazard rate is 0.78\%, which allows comparison to other studies of NPO closure.
While after very cursory search of the literature I did not find a systematic literature review or meta-analysis, a number of individual studies had investigated similar populations (see table \ref{tbl:litreview}).

\begin{table}[htbp]
\caption{\label{tbl:litreview}Hazard rates in studies of museums and other non-profits}
\centering
\begin{tabular}{llll}
\hline
study & population & avg. hazard rate & data source\\
\hline
\cite{Hager_2001_vulnerability} & Art Museums & 2.4\% & IRS\\
\cite{Gordon_etal_2013_insolvency} & Museums & 0.7\% & IRS\\
\hline
\cite{Hager_2001_vulnerability} & all Art NPOs & 2.8\% & IRS\\
\cite{Gordon_etal_2013_insolvency} & all Art NPOs & 2.08\% & IRS\\
\cite{Gordon_etal_2013_insolvency} & all NPOs & 1.58\% & IRS\\
\cite{Hager_Galaskiewicz_Larson_2007_liability} & NPOs & 1.25\% & own data\\
\cite{Clifford_2018_reinforcing} & Charities & 2\%-5\% & Reg. of Charities (UK)\\
\cite{Mayer_2022_slimmer} & NPOs & 0.17\% & IRS\\
\hline
\end{tabular}
\end{table}

\textcite{Hager_2001_vulnerability} and \textcite{Gordon_etal_2013_insolvency} both study NPOs more generally, but report survival estimates by organizational form.
At the moment it is not clear which factors account for their hazard rates to differ substantially, but it might be due to selection criteria.
Both these studies also report substantial variation in hazard rates between organizational forms.
While preliminary results indicate that private museums have dissolution rates lower or similar to museums and non-profit organizations generally, more research is needed to establish the comparability of these hazard rates.





\subsubsection*{Regression Results}


% latex table generated in R 4.3.3 by xtable 1.8-4 package
% Sun Apr  7 12:05:38 2024
\begin{table}[ht]
\centering
\begin{tabular}{p{0mm}lD{)}{)}{8)3}}
  \hline 
 \multicolumn{1}{l}{} & \multicolumn{1}{l}{Variable} & \multicolumn{1}{l}{r\_pop4}\\ 
 \hline
  \multicolumn{3}{l}{\textbf{Founder}} \\ 
 & Gender - Female & .22 \; (0.37) \\ 
   & Gender - Couple & .14 \; (0.36) \\ 
   & Founder died & .28 \; (0.53) \\ 
   \multicolumn{3}{l}{\textbf{Museum}} \\ 
 & Self-Identification - Foundation & .71 \; (0.48) \\ 
   & Self-Identification - Collection & .14 \; (0.59) \\ 
   & Self-Identification - Other & 1.32 \; (0.38)^{***} \\ 
   & AN Ranking - Included & .50 \; (0.38) \\ 
   & AN Ranking - Dropped & .82 \; (0.52) \\ 
   & Founder name in Museum name & -.35 \; (0.33) \\ 
   & MOW inclusion & -.90 \; (0.40)^{*} \\ 
   \multicolumn{3}{l}{\textbf{Environment}} \\ 
 & PM density (country) & .56 \; (0.42) \\ 
   & PM density$^{2}$ (country) & -.02 \; (0.02) \\ 
   & Pop. (millions) within 10km & .29 \; (0.11)^{**} \\ 
   & Nbr PM within 10km & .30 \; (0.09)^{***} \\ 
   & Nbr PM (10km) * Pop (10km) & -.10 \; (0.04)^{**} \\ 
   \hline
 & Museum-years & 6423 \\ 
   & Closures & 52 \\ 
   & log. Likelihood & -253.41 \\ 
   & AIC & 536.82 \\ 
   & BIC & 566.09 \\ 
   \hline 
 \multicolumn{3}{l}{\footnotesize{standard errors in parantheses.\textsuperscript{***}p $<$ 0.001;\textsuperscript{**}p $<$ 0.01;\textsuperscript{*}p $<$ 0.05.}}
\end{tabular}
\caption{Cox Proportional Hazards Regression Results} 
\label{tbl:t_reg_coxph}
\end{table}

Table \ref{tbl:t_reg_coxph} shows the results of Cox proportional hazard regression model.


\paragraph*{Control Variables}


Neither gender nor the death of the founder is associated with a significant difference in private museum closure.
While there is no particular reason to expect a gender effect, the not substantially higher likelihood for a museums to close after its founder has died indicates that private museums might be less dependent on the mood of a single person (or couple), and have in fact become institutionalized to an extent that heirs of founders feel as motivated to continue the original founders' efforts as these founders themselves.

Inclusion of founder name is not significantly associated with mortality rates.
While generally organizations whose name include the name of their founder might be more centered around the founder, and hence display higher inertia, making them less able to adapt to changing environments, these results indicate that it does not seem to be case for private museums.

Environmental variables show no association with mortality.
While the density-dependence paradigm predicts a negative effect of numbers of organizations on closing rate at low numbers (due to legitimacy, i.e. growing familiarity with a new organizational form), and a positive effect at high numbers (due to competition over resources), neither relationship is significant in the case of private museums.


\paragraph*{Identity Variables}


Museums which include the term "museum" in their name are the least likely to close, while those with the term "foundation" or "collection" have higher closure rates (but not significantly higher).
Museums which include neither of these are the most likely to close, and have mortality rates more than three times as high (\(e\)\textsuperscript{1.18} = 3.25, p<0.01) than institutions including the term "museum".
The usage of unconventional names might reflect adherence to museum standards which makes museums more understandable and familiar to third parties (e.g. audiences, other museums, companies or the government) and thereby facilitates interactions which are beneficial for long-term survival (e.g. attracting visitors, cooperations, sponsorships or subsidies).
However, as founders can choose the name of their museum arbitrarily, they might choose a name without reference to organizational forms associated with perpetuity precisely because their intention is not an institution enduring in perpetuity but rather a more temporally bounded project.




Museums included in the Museum of the World Database are \(e\)\textsuperscript{-0.88} = 0.41 (p < 0.05) times as likely to close than museums which are not, yet this result might be produced by different mechanisms.
On the one hand, it might reflect that larger and richer institutions are both more likely to be noticed by the MOW and more likely to survive (MOW inclusion would then be a proxy for resources).
On the other hand, similar to unconventional naming patterns, lack of inclusion in the MOW could be an indicator of deviance from museum standards, which could increase the unfamiliarity of the museum vis-a-vis third parties, reducing the chance of survival-enhancing interactions.




The reputation of the founder is related to closure to some extent. 
Museums whose founders are not included in the ARTnews collector ranking (reference category) are the least likely to close, followed by museums whose founders are included (not statistically significant) and by founders who were included at some point but are no longer included (statistically significant).
While the order of the categories does not fully correspond to the expected order, according to which museums whose collectors are not included would be more likely to close than collectors who are, it nevertheless provides some evidence that status matters; as in particular the loss of status appears to be substantially associated with museum closure.

\subsection*{Conclusion}

The conducted analysis shows a substantial association between private museum identity and closures.
While alternative explanations cannot be completely ruled out, a clearly understandable identity and a favorable founder reputation emerge as important predictors of the chance of a private museum to close.
Future research can expand on these findings by in more detail investigating the resource perspective of private museums, the intransparency of which has so far aggravated insights about budgetary aspects, in particular how these financial and material characteristics are moderated or mediated by processes of organizational identity.




\begin{sloppypar}
\printbibliography
\end{sloppypar}
\end{document}
