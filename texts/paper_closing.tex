% Created 2024-05-03 vr 10:56
% Intended LaTeX compiler: pdflatex
\documentclass[12pt]{article}

\usepackage[hyphens]{url}                
\usepackage{hyperref}
\usepackage[hyphenbreaks]{breakurl}
\usepackage{rotating}
\usepackage{wrapfig}
\usepackage{pdflscape}
\usepackage{fixltx2e}
\usepackage{graphicx}
\usepackage{amsmath}
\usepackage{amsfonts}
\usepackage[section]{placeins}
\usepackage{dirtree}
\usepackage{siunitx}
\usepackage{afterpage}
\usepackage{pdflscape}
\usepackage{svg}


\usepackage{booktabs}
\usepackage{dcolumn}

\usepackage{bibentry}

\sisetup{detect-all}

\sloppy
\usepackage{scalerel,stackengine}

\stackMath
\newcommand\reallywidehat[1]{%
\savestack{\tmpbox}{\stretchto{%
  \scaleto{%
    \scalerel*[\widthof{\ensuremath{#1}}]{\kern-.6pt\bigwedge\kern-.6pt}%
    {\rule[-\textheight/2]{1ex}{\textheight}}%WIDTH-LIMITED BIG WEDGE
  }{\textheight}% 
}{0.5ex}}%
\stackon[1pt]{#1}{\tmpbox}%
}

\usepackage{caption}
\usepackage[draft]{todonotes}

\captionsetup{skip=0pt}
\usepackage[utf8]{inputenc}
\usepackage[style=apa, backend=biber]{biblatex} 
\usepackage[english, american]{babel}
\DeclareLanguageMapping{american}{american-apa}
\DeclareFieldFormat{apacase}{#1}

\usepackage[T1]{fontenc}
\usepackage{csquotes}

\addbibresource{/home/johannes/Dropbox/references.bib}
\addbibresource{/home/johannes/Dropbox/references2.bib}

\usepackage{floatrow}

\usepackage{listings}
\usepackage{xcolor}
\usepackage{colortbl}

\lstset{
  language=R,                    
  basicstyle=\footnotesize,      
  numbers=left,                  
  numberstyle=\tiny\color{gray}, 
  stepnumber=1,                  
  numbersep=5pt,                 
  backgroundcolor=\color{white}, 
  showspaces=false,              
  showstringspaces=false,        
  showtabs=false,                
  frame=single,                  
  rulecolor=\color{black},       
  tabsize=2,                     
  captionpos=b,                  
  breaklines=true,               
  breakatwhitespace=false,       
  title=\lstname,                
  keywordstyle=\color{red},     
  commentstyle=\color{blue},  
  stringstyle=\color{violet},     
  escapeinside={\%*}{*)},        
  morekeywords={*,...}           
} 




% \usepackage{crimson}
% \usepackage{microtype}

% \usepackage{helvet}
% \renewcommand{\familydefault}{\sfdefault}

\usepackage{tgtermes} % times font


\usepackage{fancyhdr}
\usepackage{setspace}
\onehalfspacing
\usepackage{longtable}
\usepackage{subfig}
% \usepackage[a4paper, total={18cm, 24cm}]{geometry}
\usepackage[a4paper, margin=2.5cm]{geometry}

\pagestyle{fancy}
\fancyhf{}
\renewcommand{\headrulewidth}{0pt}
\renewcommand{\maketitle}{}

\usepackage{enumitem}
\setlist[itemize]{topsep=0pt,itemsep=0pt,parsep=0pt,partopsep=0pt}

\usepackage{multicol}
\setlength\multicolsep{0pt}



\newlist{propertyList}{itemize}{1}
\setlist[propertyList]{
  label=\textbullet,
  noitemsep,
  leftmargin=10pt,
  before=\begin{multicols}{3},
  after=\end{multicols}
  }

% \cfoot {Johannes Aengenheyster}
\rfoot {\thepage}

\listfiles

\setlength{\parindent}{1.2cm}
\author{Johannes }
\date{\today}
\title{}
\hypersetup{
 pdfauthor={Johannes },
 pdftitle={},
 pdfkeywords={},
 pdfsubject={},
 pdfcreator={Emacs 29.2 (Org mode 9.6.26)}, 
 pdflang={English}}
\begin{document}




\section*{Identity and the closure of private art museums}



\subsection*{Introduction}


In recent years, private collector-run museums have seen an unprecedented proliferation, with hundreds of organizations founded since the beginning of the millenium \parencite{Velthuis_etal_2023_boom,LarrysList_2015_report}.
These museums, run by individual collectors, constitute an addition to established organizational forms of art provision such as museums operated and financed by the state, foundations or associations.
While the emergence of this organizational form has attracted attention (cf. \cite{Kolbe_etal_2022_privatemuseum} for a literature review), the process of closure has received considerably less attention.
In this paper, I address this gap by investigating the closure event of members of this new organizational form.
Analysis which factors are associated with museum closures provides insight not only into the event of closing, but also into the ongoing processes of a private museums, such as dependency relationships.
Furthermore, analyzing which museums survive allows to make informed speculations on the future of this recently emerged organizational form.




\subsection*{private museum 3 crisis}

Museums, including private museums, might be relatively resilient \cite{Geller_Salamon_2010_resilience}: 
tightening belt, fundraising, entrepreneurial strategies, downsizing
The form of museums also implies commitment of resources, larger than that of other NPOs, which in turn provides capabilities, e.g. staff that is able to search for alternative revenue sources.





\subsection*{Private museums relationships}

A private museum is entangled in a number of dependencies.
A crucial one is the relationship to the founder, without whom the private museum would not exist.
Founders often control some aspects of management/operations.
While the exact power relationships within museums are hard to observe due to the lack of transparency, which makes it in particular to assess the role of third parties, it is plausible to assume that they cannot exist against the wishes of the founder.
Next to exhibiting the founder's collection, they likely also to a large extent dependent on continued financial contributions of the founder, as museums are chronically unprofitable (which is evidenced by their high share of NPOs or publicly owned/run museums, rather than for-profit companies). 
While the lack of transparency again makes it hard to systematically assess the financial importance of founder contributions to the existence of museums, \textbf{anecdotal} evidence suggests it to be substantial \parencite{Velthuis_Gera_forthcoming_fragility}.






Furthermore, private museums, as (non-profit) organizations, is deeply entangled with its environment.
Its ability to attract audiences, which may involve competition with other museums, secure cooperations with other museums, acquire sponsorships from cooperations and fundings from governments and other foundations are likely factors that contribute to its viability and hence longevity.
While some of these aspects can be influenced by the founder (in particular the available audience, via being able to decide a location and the extent to which a museum should organize exhibitions), other are less so.
In particular when they rely on the decision-making of other art-field actors, such as competition with other museums and sponsorship/funding choices, the influence of the individual founder is severely limited.
The image that these parties have of the private museum might be substantial importance in guiding their decision-making.



\subsection*{Private museums}


Investigating the survival prospects of private museum furthermore illucidates the long-term development of the organizational form as a whole.
While the composition of private museums is not only influenced by closures, but also by foundings, analysing the former provides insights in the sustainability of

Given the non-significant gender difference, it can be expected that (unless given a strong increase in founding by women) the population of private museums will continue to remain a dominated affair\footnote{The point estimates, which are higher for women and couples, albeit not significantly so, even point towards a slight expansion of the proportion of museums established by men.}
The role of the founder is two sided.
Death and exclusion from a prestigious ranking are not associated with a higher risk of closure, which indicates that private museums have attained a substantial degree of institutionalization.
On the other hand, self-identifying as an institution which is not a museum, foundation or collection possibly reflects a kind of founder attitudes which substantially influences survival chances.
Founders who are less interested in establish institutions might be more inclined to choose a name which less reflects this ambition. 
One way of conceptualizing this distinction might then be that whereas involuntary, externally imposed events are less able to close museums of otherwise motivated founders, some founders might from the very beginning voluntarily decide to not establish a museum that is intended for perpetuity.


However, this group is somewhat small.
Of the XXX museums included in the database, XXX do not carry the term museum, foundation or collection.
Thus given this low share, and the corresponding  higher closing rate, a possible future for the private museum population might be increased institutionalization and consolidation of relatively robust organizations.
The effects of location might be seen
On the one hand, museum location is chosen by the founder at the time of founding (or rather, several years in advance).
This gives founders substantial choice in the size and composition of the potential audience who will be able to easily assess the museum.
On the other hand, collectors are less able to influence what plans other collectors might have.
What may have been at one point been identified as a promising location without substantial competition may turn out to be a much less fertile ground as soon as other collectors start eyeing the same region for their own private museum projects.
In particular private museum in rural locations with low population densities seem to be at risk when other museums open in the neighborhood. 
This is not to say that private museums cannot endure in areas with low population densities. 
In fact, at low numbers of private museums in the neighborhood their survival prospects are better than those located in larger cities.
These larger cities with low private museum counts are possibly also characterized by lower numbers of other cultural institutions as well as an audience less interested in postmaterial art consumption.


The lack of a relation between temporary exhibitions and survival prospect is somewhat unexpected.
It was expected that as temporary exhibitions constitute one of the main tasks of a museums operations, organizing them would benefit a museum financially (via visitor fees) and symbolically (via prestige of other art field actors, which might be developed into further contacts), leading to higher survival chances\footnote{Furthermore, the ability to organization exhibitions likely reflects museum resources, which should also enhance survival.}
However, organizing exhibitions does not seem to bestow benefits for the museums which organize them, nor do organizations seem to be organized only by established/sustainable/secure organizations.
One possibility for this discrepancy could be that exhibitions are expensive, and whatever benefits gained from them are offset by the associated costs.
One might see in this association again the importance of the founder:
Founders who are committed to their museums will keep them alive, regardless of whether or not they plan to organize temporary exhibitions.








\subsection*{Theoretical framework}


As a substantial literature on organizational survival exists, 


\subsubsection*{Identity and the survival prospects of non-profit organizations}

An identity that is legitimate (understandable) and favorable (positive) has been shown to have a vast number of positive outcomes for organizations \parencite{Lange_Lee_Dai_2010_reputation}.
As gaining the approval of different stakeholders is necessary for being understood and/or supported by other organizations in the environment, having a clearly defined and favorable identity influences resource acquisition and thereby survival prospects \parencite{Rao_1994_reputation}.
Even for non-profit organizations such as museums, which do not produce tangible goods (a domain where identity-related effects are present, e.g. \cite{Hsu_2015_granted,Bogaert_etal_2014_ecological}), a clear and positive identity might be helpful in a number of ways:
In particular in the context of private museums, decline of legitimacy or reputation could lead to lower visitor numbers, less discounts from art dealers, less government grants or lower chances on collaborations with other museums or corporations, resulting in lower revenues and higher acquisition costs.


The empirical record of the link between identity and survival for non-profit organizations is mixed: 
For example \textcite{Bielefeld_1994_survival} finds that NPOs who pursue less legitimation strategies (obtaining endorsements, lobbying or contributing to local causes) are more likely to close.
However, other studies find limited impact of legitimacy on survival, both when measured via density \parencite{Bogaert_etal_2014_ecological} or from archival sources and interviews \parencite{Fernandez_2007_dissolution}.
Legitimacy might be obtained by isomorphism, i.e. adopting features associated with blueprints \parencite{diMaggio_1983_iron}.
Next to "hard" organizational features, adherence to naming conventions has been found to enhance legitimacy \parencite{Glynn_Abzug_2002_names}; organizations that adhere to field norms about name length, name ambiguity (usage of artificial names) and name domain specificity (mentioning industry) were judged more legitimate. 
However, the category of museums could be relatively flexible (for example, it is generally not subject to state regulation), which might result in a relatively high tolerance of diversity as "anything goes" and hence limited "devaluation" of non-conforming, atypical members \parencite{Bogaert_etal_2014_ecological}.


Furthermore, identity of a private museum may be influenced by the identity of persons most closely affiliated with it, in particular the identity of its founders.
Given that founders are the very reason behind the existence of these organizations, and sometimes plan these as their personal legacies \parencite{Walker_2019_collector}, founder status gains or losses may very well reflect on the wider organization and affect its survival propspects.


\bigbreak
\noindent
\textbf{Hypothesis 1}: Private museums with identities more understandable, legitimate and favorable are less likely to close. 



\subsubsection*{Organizational transformation}

\textcite{Carroll_Khessina_2019_demography} argue that transformation of core features of organizations such as changes in technology or authority can have divergent consequences. 
On the one hand, it can disrupt internal routines as well as external customer relations by decreasing (perceived) reliability and accountability as the change upsets established perceptions, which can lead to higher mortality. 
On the other hand, organizational transformation can be necessary to adapt to a changing environment, and hence be beneficial for survival.



In the case of private museums, the death of the founder might constitute a substantial transformation as authority has to be reconfigured.
Abandonment by insiders (such as divorce of directors) has been argued to contribute to NPO closure \parencite{Duckles_Hager_Galaskiewicz_2005_close}.\footnote{Conflict is similarly argued to contribute to closure but is not testable with the current data.}
Even if plans have been made for a handover, the new leaders might not share the same commitment to art as the original founder, potentially decreasing museum sustainability. 
In the case of private museums, founder death has been speculated to pose a challenge to their sustainability as "seldom do heirs share a similar passion or wish to take on the financial burden of maintaining private museum indefinitely" \parencite[p.234]{Walker_2019_collector}.
However, existing research has not found a straightforward effect of founder death on museum closure \parencite{Velthuis_Gera_forthcoming_fragility,Velthuis_etal_2023_boom}.
Nevertheless as previous research has relied primarily on descriptive statistics, investigating founder death in a multivariate survival model allows to investigate the effect of founder death with higher precision.
\bigbreak
\noindent
\textbf{Hypothesis 2}: A private museum is more likely to close after the death of the founder.



\subsubsection*{Competition}


needs to write out: 
\begin{itemize}
\item \cite{Hager_1999_demise}:
\begin{itemize}
\item p.173: Ten organizations cited an inability to compete as a proximate cause for closure, making it the most popular proximate cause of organizational closure.
\item p.173: Both [art organizations] noted that the majority of the community’s arts dollars, both individual donations/subscriptions and community and corporate grants, tended to go to the singular, most reputable arts organizations of their types (e.g. museums, orchestras, dance troupes) in the metro area. Other arts organizations faced strong competitive pressures from these flagship organizations, a factor which caused the demise of Organizations 276 and 293
\end{itemize}
\item \cite{Lecy_2010_nonprofit}:
\begin{itemize}
\item p.27: The analysis shows that a period of rapid sector expansion – a period where many additional resources are being poured into the sector – creates a more competitive environment as evidenced by higher rates of organizational mortality.
\end{itemize}
\item \cite{Park_Shon_Lu_2021_mortality}
\begin{itemize}
\item p.844: Furthermore, nonprofits that run their business in a county with greater organizational density and population size are more likely to dissolve, which implies that competition, as well as demand for service, prevents nonprofits from surviving
\end{itemize}
\item \cite{HernandezOrtiz_2022_discontinuity}
\begin{itemize}
\item p.116:  A crowded niche increased the competition and limited their access to resources and clients. As reported by informant of organization \#C07: "Well, there was less money to go around and more organizations seeking the money. Here we have the same, the same group of people that were supporting the organization, you know, donating to it and that only goes so far. So, we had a full-time grant writer, we were always applying for grants. But there are so many organizations applying for the same grants". (P31, Pos. 49-51)
\end{itemize}
\item \cite{Lu_Shon_Zhang_2019_dissolution}
\begin{itemize}
\item p.43: For example, nonprofits operating in a more competitive environment with a higher degree of organizational density would experience a higher chance of dissolution.
\end{itemize}
\item \cite{Haugh_etal_2021_nascent}:
\begin{itemize}
\item p.1227: SEs are likely to compete with non-profit organizations for grants,donations, and programme-related investments. Second,both SEs and non-profit organizations serve the disadvantaged and socially excluded and thus may also compete for service users. Although collaboration between SEs and charities could increase the supply of services to those in need (Smiddy 2010), they compete with each other for resources and beneficiaries. The more similar the resource needs and markets served, the greater the competition between the organizational forms (Baum and Haveman 1997). Hence, the population density ofregistered non-profit organizations will thus have a detrimental effect on SE survival.
\end{itemize}
\end{itemize}

resource-based: lack of audience, or many other people competing for audience
\cite{Velthuis_Gera_forthcoming_fragility}:
p.23: In other cases, the location of the museum is a factor in limited visitor interest. On the
one hand, private museums located in metropolitan art centers tend to face competition over
visitors  from  well-established  other  museums  in  the  vicinity.  On  the  other  hand,  private
museums located outside of those centers can be confronted with a small local audience, and
difficulties in attracting visitors from afar.

\textbf{Hypothesis 3}: A private museum is more likely to close in a more competitive environment.



\subsubsection*{Other predictors of organizational survival}

\paragraph*{Age}



The age of an organization is considered in different ways to be related to closing.
What started with liability of newness \parencite{Stinchcombe_1965_structure}, which posited that organizational mortality decreases over time due to acquisition of capabilities and connections, has lead to a number of extensions \parencite{Carroll_Khessina_2019_demography,Hannan_1998_mortality}:
Liability of adolescence argues that shortly after foundings organizations can still draw on endowments, the expiration of which leads to a peak of mortality some time after opening.
While according to this view mortality still decreases after the peak as the organization becomes more established, other perspectives argue also for a positive relationship between age and mortality:
According to the liability of obsolescence paradigm the environment changes faster than organizations which are seen as relatively constrained by inertia, which therefore leads to declining fitness and survival chances.
Similar, according to "liability of senescence" capabilities/routines/coalitions can constrain the actions of organizations.
Given such conflicting predictions, \textcite{Carroll_Khessina_2019_demography} propose to model age dependence with piece-wise constant hazard models (rather than parametric models) to estimate age-specific effects which can, but do not have to, indicate a relationship between age and mortality.




\subsection*{Data and Methods}


\subsubsection*{Dependent Variable}

A newly constructed database, assembled from different sources,  is used to measure the survival of private museums (cf. \textcite{Velthuis_etal_2023_boom} for details).
First, existing databases of private art museums are harmonized.
Secondly, the online presences of 14 English art publications were exhaustively searched for any mention of private museums.
After identifying unique institutions, data on these organizations was collected such as opening year, closing year (of those which closed) and founder information.
Currently, the database includes 547 museums (of which 446 are currently open, the remainder having closed or transformed into other organizational forms).
Using this database allows to reconstruct the lifecourse of the each institution with the the unit of analysis as museum-year:
The main dependenent variable is thus an indicator of museum closure, which takes the value of 1 for the museums which close in the year of their closure, and 0 otherwise.
Typically for survival data, museums that do not close during the observation period are right-censored, i.e. for such museums the dependent variable is zero for all years.


\subsubsection*{Main Predictors}



\paragraph*{Legitimacy}


% latex table generated in R 4.3.3 by xtable 1.8-4 package
% Fri May  3 10:55:27 2024
\begin{table}[ht]
\centering
\begin{tabular}{lrll}
  \hline 
 \multicolumn{1}{l}{Self-identification} & \multicolumn{1}{l}{N} & \multicolumn{1}{l}{Self-ID (recoded)} & \multicolumn{1}{l}{examples}\\ 
 \hline
 Museum & 190 & Museum & Museum Kampa, Museum Barberini \\ 
  Foundation & 104 & Foundation & Marciano Art Foundation, Hill Art Foundation \\ 
  Collection &  69 & Collection & The Farjam Collection, Sammlung Boros \\ 
  None &  64 & Other & The Bunker, The Broad \\ 
  Center &  17 & Other & Dairy Art Centre, Art Center Nabi \\ 
  Gallery &  14 & Other & Saatchi Gallery, Galerie C15 \\ 
  Art space &  13 & Other & El Espacio 23, Qiao Space \\ 
  House / villa &  11 & Other & Villa La Fleur, Casa Daros Rio \\ 
  Institute &  10 & Other & Instituto Inhotim, Woods Art Institute \\ 
  Kunsthalle &   8 & Other & Kunsthalle Würth, G2 Kunsthalle \\ 
  Park / garden &   3 & Other & Schlosspark Eyebesfeld, Il Giardino dei Lauri \\ 
   \hline
\end{tabular}
\caption{Selfidentification} 
\label{tbl:t_selfid}
\end{table}

Two measures are constructed that aim to capture an effect of legitimacy on closing.
First, self-identification measures whether a the name of a museum includes the term "Museum", "Foundation", "Collection", or other (such as "Villa", "Institute", "Center" or "Gallery", which are grouped together as they are less frequent, see table \ref{tbl:t_selfid}).
Since the name is one of the first aspects that stakeholders perceive when engaging with an organization, it might very well shape expectations and influence forms of engagement.
In particular, it might be that museums that also refer to themselves as museums are seen as more favorable and worthy of visits, collaborations and support than other museums.


Secondly, a binary predictor measures whether the museum is included in the "Museums of the World" (MOW) database \parencite{deGruyter_2021_MOW}, a database containing over 55.000 museums.
While the exact criteria of the research process leading to the MOW database are not known, it is plausible that museums who correspond to an ideal-typical museum are more likely to be included.
Assuming that museums are seen similarly by the creators of the MOW database and other audiences, inclusion in the MOW database constitutes a proxy for adherence to museum standards. 


\paragraph*{Reputation}


Reputation is measured via inclusion of the founder in the Artnews top 200 collector ranking \parencite{Artnews_ranking}.
This ranking, established in 1990, lists each year the 200 collectors the editors of the magazine consider most relevant.
For each year, a museums can have one of three values: "Not included"; in which case the founder is not included in the year in question and has not been so in the past, "Included"; when the founder is included, or "Dropped", for cases where the founder was included at some point in the past but is no longer included in the current year.
Cases where multiple individuals are involved in the founding of a private museum are resolved by aggregating the Artnews ranking inclusion history to the founder level (i.e., if only one of two founders is dropped from the Artnews ranking but the other remains included, the founder couple as a whole is considered included).
The variable is lagged by one year to avoid reverse causality, i.e. to exclude the possibility that an association is observed which is present due to founders being dropped from the ranking as a consequence of the closure of their museum.


\paragraph*{Organizational transformation}

A measure of founder death is constructed as a time-varying covariate, and is set to 1 for museum-years after the death of the founder. 

\paragraph*{Competition}

On a national level, density of private museum is calculated as the number of private museums per million.
On a local level, I use the Global Human Settlement Layer \parencite{EC_2023_GHSL} to estimate the number of private museums and population counts inside a radius of ten kilometers around each private museum for each year.
To be able to estimate both competition and demand/audience/population separately, I do not calculate densities as per capita rates, but include the local museum counts, local population in millions as well as an interaction between the two.

\subsubsection*{Control variables}


\bigbreak
\noindent
A number of control variables are used: 
On the level of the founder, gender constitutes a time-invariant variable; with possible values being male, female and couple.
Centrality of the founder's vision is measured via the proxy of whether the name of the founder is included in the name of the museum.








\subsubsection*{Analytical framework}



First, I investigate the relationship between age and closing descriptively using hazard rates and the Kaplan-Meier survival function.
This non-parametric method estimates the hazard rate as the conditional probability of exiting at time t\textsubscript{i} given being alive as \(h(t_i) = \frac{d_i}{n_i}\), (with \(n_i\) as the number of units at risk at time \(t_i\) and \(d_i\) as the number of exits at time \(t_i\)), and the survival function as the product of these conditional probabilities as \(S(t) = \prod_{t_i \geq t} \left(1-h(t_i) \right)\).



Secondly, we use Cox proportional hazard models which estimate the relationship between covariates and the hazard rate.
The Cox-proportional hazard model has the form \(h(t,\mathbf{x}) = h_0(t) \psi\), where \(\psi = \exp(\sum_{j} \mathbf{x}_j \beta_j)\); with \(h(t)\) as the hazard rate, \(h_0(t)\) as the baseline hazard rate, \(\mathbf{x}\) as the set of variables indexed by \(j\), and \(\beta\) as the coefficient vector.
Partial maximum likelihood estimation allows calculating \(\beta\) without defining the baseline hazard rate, which therefore is not constrained to a parametric form. 
Coefficients are interpeted as log hazard ratios, i.e. as multipliers of the baseline hazard rate. 



\subsection*{Results}


\subsubsection*{Summary statistics}


% latex table generated in R 4.3.3 by xtable 1.8-4 package
% Fri May  3 10:55:26 2024
\begin{table}[ht]
\centering
\begin{tabular}{llrrrrrr}
  \hline
 & & \multicolumn{2}{c}{Museum} & \multicolumn{4}{c}{Museum-year} \\ 
\cmidrule(r){3-4}\cmidrule(r){5-8} \multicolumn{1}{l}{} & \multicolumn{1}{l}{Variable} & \multicolumn{1}{l}{Count} & \multicolumn{1}{l}{Mean} & \multicolumn{1}{l}{Mean} & \multicolumn{1}{l}{SD} & \multicolumn{1}{l}{Min.} & \multicolumn{1}{l}{Max.}\\ 
 \hline
  \multicolumn{8}{l}{\textbf{Founder}} \\ 
 & Gender - Male & 294 & 0.584 & 6.0e-01 &   0.49 & 0 & 1 \\ 
   & Gender - Female & 79 & 0.157 & 1.7e-01 &   0.37 & 0 & 1 \\ 
   & Gender - Couple & 130 & 0.258 & 2.3e-01 &   0.42 & 0 & 1 \\ 
   & Founder - alive & 440 & 0.875 & 9.0e-01 &   0.30 & 0 & 1 \\ 
   & Founder - died recently & 17 & 0.034 & 1.9e-02 &   0.13 & 0 & 1 \\ 
   & Founder - died 2+ years ago & 46 & 0.091 & 7.9e-02 &   0.27 & 0 & 1 \\ 
   \multicolumn{8}{l}{\textbf{Museum}} \\ 
 & Self-Identification - Museum & 190 & 0.378 & 4.1e-01 &   0.49 & 0 & 1 \\ 
   & Self-Identification - Foundation & 104 & 0.207 & 1.9e-01 &   0.39 & 0 & 1 \\ 
   & Self-Identification - Collection & 69 & 0.137 & 1.3e-01 &   0.34 & 0 & 1 \\ 
   & Self-Identification - Other & 140 & 0.278 & 2.7e-01 &   0.45 & 0 & 1 \\ 
   & Founder name in Museum name & 240 & 0.477 & 4.8e-01 &   0.50 & 0 & 1 \\ 
   & MOW inclusion & 92 & 0.183 & 2.7e-01 &   0.45 & 0 & 1 \\ 
   & AN Ranking - Not Included &  &  & 8.0e-01 &   0.40 & 0 & 1 \\ 
   & AN Ranking - Included &  &  & 1.4e-01 &   0.35 & 0 & 1 \\ 
   & AN Ranking - Dropped &  &  & 5.3e-02 &   0.23 & 0 & 1 \\ 
   & Exhibition any &  &  & 6.3e-01 &   0.48 & 0 & 1 \\ 
   & Exhibition any last 5 years &  &  & 4.6e-01 &   0.50 & 0 & 1 \\ 
   & PC1 (Size) & 112.930441195886 & 0.225 & 1.8e-01 &   1.90 & -6.5 & 2.61 \\ 
   & PC2 (Support) & 90.555299883145 & 0.180 & 2.0e-01 &   1.52 & -5.1 & 2.71 \\ 
   \multicolumn{8}{l}{\textbf{Environment}} \\ 
 & PM density (country) &  &  & 4.5e-01 &   1.26 & 0.00077 & 27.26 \\ 
   & Pop. (millions) within 10km &  &  & 1.3e+00 &   1.68 & 0.000087 & 10.81 \\ 
   & Pop. (millions) country &  &  & 1.6e+02 & 292.68 & 0.037 & 1412.36 \\ 
   & Nbr PM within 10km &  &  & 1.8e+00 &   3.14 & 0 & 16 \\ 
   & PM density (10km) &  &  & 1.7e+00 &   4.93 & 0 & 76.37 \\ 
   & Europe and North America & 320 & 0.636 & 6.4e-01 &   0.48 & 0 & 1 \\ 
   & Region - Africa & 8 & 0.016 & 1.2e-02 &   0.11 & 0 & 1 \\ 
   & Region - Asia & 142 & 0.282 & 2.9e-01 &   0.45 & 0 & 1 \\ 
   & Region - Europe & 248 & 0.493 & 5.0e-01 &   0.50 & 0 & 1 \\ 
   & Region - Latin America & 23 & 0.046 & 4.2e-02 &   0.20 & 0 & 1 \\ 
   & Region - North America & 72 & 0.143 & 1.4e-01 &   0.34 & 0 & 1 \\ 
   & Region - Oceania & 10 & 0.020 & 1.9e-02 &   0.13 & 0 & 1 \\ 
   & year &  &  & 2.0e+03 &   5.72 & 2000 & 2021 \\ 
   & Time Period 2000-2004 &  &  & 1.0e-01 &   0.30 & 0 & 1 \\ 
   & Time Period 2005-2009 &  &  & 1.7e-01 &   0.37 & 0 & 1 \\ 
   & Time Period 2010-2014 &  &  & 2.6e-01 &   0.44 & 0 & 1 \\ 
   & Time Period 2015-2019 &  &  & 3.3e-01 &   0.47 & 0 & 1 \\ 
   & Time Period 2020-2024 &  &  & 1.4e-01 &   0.35 & 0 & 1 \\ 
   & Great Recession (2008/09) &  &  & 7.6e-02 &   0.27 & 0 & 1 \\ 
   & Covid Pandemic (2020/21) &  &  & 1.4e-01 &   0.35 & 0 & 1 \\ 
   \hline
\end{tabular}
\caption{Summary Statistics} 
\label{tbl:t_sumstats}
\end{table}

In total, 479 private museums are included in the database, 53 of them have closed.
6903 museum-years are observed; table \ref{tbl:t_sumstats} shows summary statistics for all variables.

\subsubsection*{Hazard rate and Age Dependence}


\begin{figure}[htbp]
\centering
\includegraphics[width=16cm]{../figures/p_hazard.pdf}
\caption{\label{fig:p_hazard}Private Museum hazard function}
\end{figure}

\begin{figure}[htbp]
\centering
\includegraphics[width=14cm]{../figures/p_surv.pdf}
\caption{\label{fig:p_surv}Private Museum Survival probability}
\end{figure}


Figure \ref{fig:p_hazard} shows the hazard rate over time, figure \ref{fig:p_surv} the corresponding survival probability.
The hazard rate doubles over the first 10 years from approximately 0.45\% to 0.9\%, at which value it approximately stays constant for the next 20 years, after which it appears to decline.
However, the decline after thirty years is presumably less certain as very few private museums have already reached this age.


The increase in the first years is likely due to endowments: As the establishment of a museum requires considerable resources, their founders appear to allocate enough resources to keep the risk of closure small in the first years after opening.
Given that founders are likely aware that their endeavor is unlikely to generate a profit, they are not vulnerable to the liability of newness \parencite{Stinchcombe_1965_structure}.
The relative stability of the hazard rate after ten years has no correspondence in classical theories on age dependence:
While an increase in mortality as observed here is predicted by the idea of "liability of adolescence" \parencite{Carroll_Khessina_2019_demography}, this framework also predicts a decline in mortality after a peak due to becoming established.
Furthermore, while "liability of obsolescence" and "liability of senescence" (ibid.) predict higher mortality for older organizations due to mismatch with the environment and inertia (inflexible internal routines), it seems questionable whether museums would have reached obsolescence or senescence already after ten years.
Furthermore, the applicability of obsolescence and senescence might be limited for non-profit organizations.


Overall the entire life span, the average hazard rate is 0.87\%, which allows comparison to other studies of NPO closure.
While I did not find a systematic literature review or meta-analysis, a number of individual studies had investigated similar populations based on IRS data (see table \ref{tbl:litreview}) 

\begin{table}[htbp]
\caption{\label{tbl:litreview}Hazard rates in studies of museums and art NPOs}
\centering
\begin{tabular}{lll}
\hline
study & population & avg. hazard rate\\
\hline
\cite{Hager_2001_vulnerability} & Art Museums & 2.4\%\\
\cite{Gordon_etal_2013_insolvency} & Museums & 0.7\%\\
\cite{Bowen_1994_charitable} & Museums & 1.1\%\\
\hline
\cite{Hager_2001_vulnerability} & all Art NPOs & 2.8\%\\
\cite{Gordon_etal_2013_insolvency} & all Art NPOs & 2.08\%\\
\hline
\end{tabular}
\end{table}

\textcite{Hager_2001_vulnerability} and \textcite{Gordon_etal_2013_insolvency} both study NPOs more generally, but report survival estimates by organizational form.
Both these studies also report substantial variation in hazard rates between organizational forms.
While these comparisons indicate that private museums have dissolution rates lower or similar to museums and non-profit organizations generally, more research is needed to establish the comparability of these hazard rates.


Comparison to other forms of nonprofit organizations reported by \parencite{Gordon_etal_2013_insolvency}
places private museum at a risk similar to that of animal shelters and zoos (0.7\%), botanical \& environmental centers (0.8\%), colleges and universities (0.8\%), food and agricultural-oriented human service organizations (0.8\%) and  student and educational services (0.8\%).
While the diversity of non-profit organizations makes comparisons to its entirety difficult, it is still worthwhile to consider that the average hazard of private museums 0.87\% is around half that of that NPOs generally (1.58\%, see table \ref{tbl:litreview2} for studies on wider nonprofit populations).
It is furthermore substantially lower than that of youth centers (1.4\%), children-focused human services (2.4\%), K12 schools (1.8\%), mental health organizations (2.4\%) or performing arts organizations (3.7\%).
In fact, lower dissolution rates are only reported for historical societies (0.5\%) and libraries (0.2\%).



\begin{table}[htbp]
\caption{\label{tbl:litreview2}Hazard rates in studies of NPOs}
\centering
\begin{tabular}{llll}
\hline
study & population & avg. hazard rate & data source\\
\hline
\cite{Gordon_etal_2013_insolvency} & NPOs & 1.58\% & IRS\\
\cite{Hager_Galaskiewicz_Larson_2007_liability} & NPOs & 1.25\% & own data\\
\cite{Clifford_2018_reinforcing} & Charities & 2\%-5\% & Reg. of Charities (UK)\\
\cite{Mayer_2022_simmer} & public charities & 0.17\% & IRS\\
\hline
\end{tabular}
\end{table}





\subsubsection*{Regression Results}


% latex table generated in R 4.3.3 by xtable 1.8-4 package
% Fri May  3 10:55:34 2024
\begin{table}[ht]
\centering
\begin{tabular}{p{0mm}lD{)}{)}{8)3}}
  \hline 
 \multicolumn{1}{l}{} & \multicolumn{1}{l}{Variable} & \multicolumn{1}{l}{r\_pop4}\\ 
 \hline
  \multicolumn{3}{l}{\textbf{Founder}} \\ 
 & Gender - Female & .02 \; (0.34) \\ 
   & Gender - Couple & -.10 \; (0.32) \\ 
   & Founder - died recently & 1.09 \; (0.61) \\ 
   & Founder - died 2+ years ago & -.06 \; (0.55) \\ 
   \multicolumn{3}{l}{\textbf{Museum}} \\ 
 & Self-Identification - Foundation & .25 \; (0.42) \\ 
   & Self-Identification - Collection & -.46 \; (0.54) \\ 
   & Self-Identification - Other & 1.12 \; (0.31)^{***} \\ 
   & AN Ranking - Included & .25 \; (0.35) \\ 
   & AN Ranking - Dropped & .51 \; (0.50) \\ 
   & Founder name in Museum name & .09 \; (0.28) \\ 
   & Exhibition any & -.22 \; (0.27) \\ 
   \multicolumn{3}{l}{\textbf{Environment}} \\ 
 & PM density (country) & .63 \; (0.35) \\ 
   & PM density$^{2}$ (country) & -.03 \; (0.02) \\ 
   & Pop. (millions) within 10km & .27 \; (0.09)^{**} \\ 
   & Nbr PM within 10km & .26 \; (0.08)^{***} \\ 
   & Nbr PM (10km) * Pop (10km) & -.09 \; (0.03)^{**} \\ 
   & Great Recession (2008/09) & -1.61 \; (1.01) \\ 
   & Covid Pandemic (2020/21) & .39 \; (0.29) \\ 
   \hline
 & Museum-years & 6096 \\ 
   & Closures & 68 \\ 
   & log. Likelihood & -338.36 \\ 
   & AIC & 712.73 \\ 
   & BIC & 752.68 \\ 
   \hline 
 \multicolumn{3}{l}{\footnotesize{standard errors in parantheses.\textsuperscript{***}p $<$ 0.001;\textsuperscript{**}p $<$ 0.01;\textsuperscript{*}p $<$ 0.05.}}
\end{tabular}
\caption{Cox Proportional Hazards Regression Results} 
\label{tbl:t_reg_coxph}
\end{table}

Table \ref{tbl:t_reg_coxph} shows the results of Cox proportional hazard regression model.


\paragraph*{Control Variables}


Neither gender nor inclusion of founder name is significantly associated with mortality rates.
While generally organizations whose name include the name of their founder might be more centered around the founder, and hence display higher inertia, making them less able to adapt to changing environments, these results indicate that it does not seem to be case for private museums.


\paragraph*{Identity Variables}


Museums which include the term "museum" in their name are the least likely to close, while those with the term "foundation" or "collection" have higher closure rates (but not significantly higher).
Museums which include neither of these are the most likely to close, and have mortality rates more than three times as high (\(e\)\textsuperscript{1.18} = 3.25, p<0.01) than institutions including the term "museum".
The usage of unconventional names might reflect adherence to museum standards which makes museums more understandable and familiar to third parties (e.g. audiences, other museums, companies or the government) and thereby facilitates interactions which are beneficial for long-term survival (e.g. attracting visitors, cooperations, sponsorships or subsidies).
However, as founders can choose the name of their museum arbitrarily, they might choose a name without reference to organizational forms associated with perpetuity precisely because their intention is not an institution enduring in perpetuity but rather a more temporally bounded project.


Museums included in the Museum of the World Database are \(e\)\textsuperscript{-0.88} = 0.41 (p < 0.05) times as likely to close than museums which are not, yet this result might be produced by different mechanisms.
On the one hand, it might reflect that larger and richer institutions are both more likely to be noticed by the MOW and more likely to survive (MOW inclusion would then be a proxy for resources).
On the other hand, similar to unconventional naming patterns, lack of inclusion in the MOW could be an indicator of deviance from museum standards, which could increase the unfamiliarity of the museum vis-a-vis third parties, reducing the chance of survival-enhancing interactions.


The reputation of the founder is not statistically related to closure. 
Museums whose founders are not included in the ARTnews collector ranking (reference category) are the least likely to close, followed by museums whose founders are included (not statistically significant) and by founders who were included at some point but are no longer included (not statistically significant).

\paragraph*{Organizational transformation}


The death of the founder is not associated with a significant difference in private museum closure, i.e. museums are not significantly more likely to close after their founder has died.
This possibly indicates that private museums might be less dependent on the mood of a single person (or couple), and have in fact become institutionalized to an extent that heirs of founders feel as motivated to continue the original founders' efforts as these founders themselves.

\paragraph*{Competition}

\begin{figure}[htbp]
\centering
\includegraphics[width=18cm]{../figures/p_pred_popprxcnt.pdf}
\caption{\label{fig:p_pred_popprxcnt}Predicted Avg. Hazard Rate on Regional PM Density and Population}
\end{figure}



Country-level Environmental variables show no association with mortality.
However, at the local level the number of private museum and population density predicts survival chances as the main effects of population number of private museums as well as its interaction is significant.
Figure \ref{fig:p_pred_popprxcnt} shows predicted average hazard rates for different conditions of regional PM density and population numbers.
Museums located in sparsely populated areas have (at low PM densities) lower hazard rates than museums in more densely populated areas. 
Furthermore, these "rural" museums see an increase in the risk of closing for each additional private museum present within 10 kilometers, whereas the risk of closing decreases with  each additional private museum in the local environment for museums in more urban areas. 


\subsection*{Conclusion}

TBD




\begin{sloppypar}
\printbibliography
\end{sloppypar}



\subsubsection*{Limitations}


For this study, I only investigated the occurence of closure events.
However, private museums can stop existing as private museums not just by closing down, but also by being transformed into other organizational forms.
One way in which this can happen is when other parties, such as local governments other philanthropists, foundations, or corporations become substantially involved in the ownership, governance and/or funding of the museum.
Sometimes this involves changes in the composition of the board of directors, with new parties gaining seats and hence influence in decision-making, thereby diminishing the private character of the museum, often to an extent that it no longer fits the working definition.
Museums which experienced this trajectory have been excluded from the analysis as explorative investigation showed that it was often not feasible to determine the date of such transitions\footnote{As a consequence of this, they are excluded from the analysis alltogether, and thus also not contributing to the the hazard rate for closing, despite being entities at risk for closing. However, as their numbers are limited to 25 entries in the database, the bias this introduces into closing risks is likely not substantial.}
Primarily this was due to the intransparency of these organizations, as well as the gradual process.
As the presence of non-founder members does not categorically exclude a museum from being categorized as private museum as long as third-party influence is limited, this influence can grow until the museum bears little resemblance to its former self.
The possibility of other outcomes than surival and closures has been investigated for NPOs more generally
\parencite{Searing_2020_zombies,HernandezOrtiz_2022_discontinuity,Helmig_Ingerfurth_Pinz_2013_nonprofit}, and for private museums specifically: 
\textcite{Walker_2019_collector} argues that in the case of Germany, private museum founders turn to the state to ensure the survival of their projects in the form of public-private partnerships.
Investigating this process of transformation into a different organizational form thus provides fruitful avenue for further research. 



environment can be measured more
\begin{itemize}
\item geocoding of other institutiosn
\item tax incentives
\item government spending
\end{itemize}
-> this might futher specify the niches in which PMs can endure
\end{document}
